% Summary of the paper we implemented
In this work, we implemented the Motion graph from~\cite{kovar2002}. By constructing a 
directed graph to locate the transition between a group of motion clips, we can 
synthesize an infinite length of motion by traversing on the graph. 


However, the result comes with several Issues:
\begin{enumerate}
\item state transition: in current system, we only consider pose transition instead of state transition. 
That is, if we blending pose from "run" to "jump", it's possible to select a transition which ending pose is 
float on the ground and then landing. Therefore, the further more improvement could consider the state 
transition, ex: running to walking and then stop. 

\item Skeleton definition: with limited motion capture files, we just ignore the skeleton 
definition of each motion. Therefore, sometimes the character may looks like 
"float" on the ground. It's possible to fix skeleton definition by certain post processing 
techniques. However, it's beyond our original goal.

\item Memory usage: without appropriate threshold to limit the number of transition node, we may 
generate to many unnecessary transition between motions. This may cause out of memory during the runtime. 
\end{enumerate}