%introduce why it's important
Fighting games are very popular since the smooth animation and the energetic of martial arts. 
In most video games, the fighting skills are used to be fancy and exaggerated. 
Players expect the motion of the virtual character should be realistic and 
smoothly. Many game companies use the motion capture techniques to produce varios 
high quality motion data for the realistic purposes. 

%fighting game by motion capture
However, motion capture is limited in the precaptured motion clips and the 
variety of motion meets the user specification. 
If we can't get the appropriate or desired motion, the virtual character may not 
to perform certain actions smoothly.
Actually, creating transition between two motion is a hard animation problem.
% why it's so difficult
A basic idea is to find the similar pose between two clips, and using 
interpolation techniques to blend the motion. However, even the poses between 
motions are similar, the momentum of each bone is still different. 
It would be a challenging problem to blend the motions at certain frames.

% our goal 
Our goal is to generate the realistic motion by motion capture library while 
also giving a user the ability to control the virtual character to act smooth 
fighting actions.
%what we've done
In this work, we implemented the motion graph from [ref].
The idea of motion graph is to find the transition points between two clips, 
whose poses are similar. 
With the detecting algorithm, we can automatically find the transition points 
and reliably generate the transition with certain blending techniques.

%the experiment result
The result shows in Section4. We collect 8 motion capture files from CMU motion 
capture website[]. The result video can be seen at []. In the result, we 
generate xxx transition nodes, xxx transitions, xxx segments. 
The result shows the motions are highly interactive and smooth with realistic.

% the following paper organization:
The following of this paper is organised  as follows.
In Section 2 we describe related work. In Section 3 we describe how a motion
graph is constructed from a database of motion capture. 
InSection 4 we set forth a general framework for extracting motion from the
motion graph that meets user specifications. 
Section 5 presented the motion synthesized result. We conclude in Section 6 with a
discussion of the contribution of the work.