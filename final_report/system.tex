In this section, we explain the detail of each step in our system. The system 
flow is depicted in Fig.[xx]. 
%Fig : system flow

The remainder of this section is divided into four pars. 
First, we explain the algorithm of detecting candidate transitions between 
motion clips. The following two sections, we discuss about how to blend the 
transition, and synthesize the motion with consistent momentum.
Finally, we explain how to extend the topology to make the virtual character 
looks natural.


\subsection{Detecting Candidate Transitions}
In this section, we discuss about how to detect the transition points between two 
motion clips. 

Generate point cloud, see Fig.[]. 

from i to j frame, window size k frames.


\subsection{Motion Transition}
After detecting the transition points between two motion clips, we need to 
synthesise the transition motion from $A_i$ to $B_j$. 
Use Floyd-Warshall algorithm to construct the all-pairs shortest paths. 


%the data structure of a node:

\subsection{Consistent Momentum between Motions}
Due to we can't get the motion capture files they are consistent in the 
momentum, it's required to synthesise the motion with the same momentum or even 
exaggerate the motion to make it looks powerful in the fighting.


\subsection{Infinite Motion Graph}
In order to make the character looks alive, we introduce the rest motion to make 
the character can transit to rest motion if the user doesn't give any command.

