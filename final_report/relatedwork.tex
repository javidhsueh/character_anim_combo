% the Motion graph, done by Kovar 2002
In~\cite{kovar2002}, they present the idea of motion graph, which successfully transform the motion synthesis problem 
into selecting transition candidate notes on the motion clips with graph walks. 
By extracting the closest pose between two clips, the transition motion can be done by motion blending techniques.
The final motion can be synthesized by walking on the graph and concatenate the visited segments.

% other paper  Motion graph ++
In~\cite{min}, they introduced a generative statistical model to analyze and
synthesize a rich repertoire of human activities. 
Their models are appealing for motion analysis and synthesis because they are compact,
highly structured, contact aware, semantic embedding, and scalable to huge and heterogeneous datasets.

Another motion synthesis approach is to construct statistical models. in~\cite{pullen2000}, they presented the kernel-based 
probability distribution to generate new motion based on statistical factors.