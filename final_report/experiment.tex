In our preliminary experiment, we selected 8 motion capture files from CMU motion capture library~\cite{cmu} 
as our candidate motions, see Table.~\ref{table:motionlist}.
Constructing motion graph is the most time-consuming task. In this experiment, 
we took about 24 hours to the final construct motion graph under Intel Core i7 960 3.2GHz, RAM 12 GB, Window7 64bit.
The generated result shows in Table.~\ref{table:motionlist}. 
The result video can be watched at~\cite{video}.
In this experiment, we generated 2886 transition nodes and 5652 edges. 
The result shows our algorithm successfully generate a smooth and seamless 
synthesized motion. Most transition can be done in a third of second.

% table, list all actions, frames, 

\begin{table}[ht]
\caption{Candidate motion capture clips}
\centering
\begin{tabular}{c c c c} 
\hline\hline 
\# & Motion name & Frame\# & Transition node# \\ [0.5ex] % inserts table 
%heading
\hline % inserts single horizontal line
1 & Rest & 799 & 386 \\ % inserting body of the table
2 & Walk & 343 & 166 \\
3 & Run & 130 & 50 \\
4 & Punch & 1854 & 453 \\
5 & Boxing & 2783 & 679 \\
6 & Jump & 410 & 179 \\
7 & Slash & 2251 & 456 \\
8 & MartialArts & 3261 & 517 \\ [1ex]
\hline % inserts single horizontal line
total &  & 11531 & 2886 \\ [1ex]
\hline
\end{tabular}
\label{table:motionlist} % is used to refer this table in the text
\end{table}
